\chapter{Introduction to Protocols for Concurrency Control in Databases}% .75
\section{Two-Phase Locking Techniques for Concurrency Control}% .25
Een \textbf{lock} is een variabele geassocieerd met een data-item dat de status weergeeft wat betreft de mogelijke operaties die erop kunnen worden uitgevoerd.
\begin{itemize}
	\item \'E\'en lock per data-item in de database
	\item Gebruikt voor synchronisatie bij gelijktijdige toegang
\end{itemize}


\subsection{Types of Locks and System Lock Tables}% 3.25
Hieronder worden enkele soorten \textbf{locks} besproken.

\subsubsection{Binary Locks}
\paragraph{Status}
Variabele heeft slechts twee toestanden voor data-item \textbf{X}:
\begin{itemize}
	\item \textbf{Locked (1)}: variabele heeft waarde 1 en \textbf{X} is niet toegankelijk
	\item \textbf{Unlocked (0)}: variabele heeft waarde 0 en \textbf{X} is toegankelijk. Indien het item wordt gebruik wordt de waarde op 1 (Locked) gezet.
\end{itemize}
We refereren vanaf nu naar de status van \textbf{lock} geassocieerd met \textbf{X} als de booleaanse waarde \textbf{lock(X)}

\paragraph{Operaties} Er zijn twee operaties gedefinieerd:
\begin{itemize}
	\item \textbf{lock\_item(X)}\\  Idien een transactie \textbf{X} wil benaderen wordt deze operatie uitgevoerd. Er zijn twee mogelijkheden:
	\begin{itemize}
		\item \textbf{lock(x) = 1}: De transactie moet wachten.
		\item \textbf{lock(x) = 0}: De waarde van \textit{lock(x)} wordt op 1 gezet en de transactie kan worden uitgevoerd.
	\end{itemize}
	
	\item \textbf{unlock\_item(X)}\\  Wanneer de transactie voltooid is wordt \textbf{X} terug vrijgegeven. De waarde van \textit{lock(x)} wordt terug op 0 gezet.
\end{itemize}
(Exact algoritme zie pagina 757)

\paragraph{Critische secties} Bovenstaande operaties moeten ondeelbaar of atomair worden uitgevoerd. Dergelijke stukken code worden \textbf{critische secties} genoemd.

\paragraph{Gegevensstructuur} Voor de implementatie van een \textbf{binary lock} volstaan twee gegevensstructuren:
\begin{itemize}
	\item Een queue waarin transacties worden geplaatst die wachten om de lock te bemachtigen.
	\item  Een gegevensstructuur met 3 velden: $Data\_item\_name, LOCK, Locking\_transaction>$. \\Dit zou bijvoorbeeld ook een hash-tabel kunnen zijn waarbij er wordt vanuit gegaan dat items die niet in de tabel zitten niet gelock zijt.
\end{itemize}

\paragraph{Regels bij Binary Lock}
\begin{enumerate}
	\item Een transactie moet een operatie \textit{lock\_item(X)} uitvoeren alvorens een operatie \textit{read\_item(X)} of \textit{write\_item(X)} uit te voeren.
	\item Een transactie moet een operatie \textit{unlock\_item(x)} uitvoeren nadat alle \textit{read\_item(X)} of \textit{write\_item(X)} operaties zijn afgerond.
	\item Een transactie mag geen operatie \textit{lock\_item(X)} uitvoeren wanneer hij de lock reeds bezit.
	\item Een transactie mag geen operatie \textit{unlock\_item(X)} uitvoeren wanneer hij de lock niet bezit.
\end{enumerate}

\paragraph{Conclusie}
\begin{itemize}
	\item Zeer eenvoudig
	\item Slechts 1 transactie kan een lock vasthouden
\end{itemize}

 
\subsubsection{Shared/Exclusive (or Read/Write) Locks}
Bepaalde operaties kunnen gelijktijdig worden uitgevoerd op \textbf{X}. Daarvoor zijn er nu 3 operaties met bijhorende status:
\begin{itemize}
	\item \textbf{read\_lock(X) $\rightarrow$ read-locked} \\ Deze lock kan door meerdere transacties tegelijk worden verkregen.
	\item \textbf{write\_lock(X) $\rightarrow$ write-locked} \\ Deze lock kan exclusief aan \'e\'en transactie worden toegewezen.
	\item \textbf{unlock(X) $\rightarrow$ unlocked} \\ Geef X terug vrij.
\end{itemize}
Deze locks worden ook \textbf{multiple-mode locks} genoemd.

\paragraph{Gegevensstructuur}
We moeten nu bijhouden hoeveel transacties in het bezit zijn van een \textbf{read lock}.
\[<Data\_item\_name, LOCK, No\_of\_reads, Locking\_transaction(s)>\]
Afhankelijk van het type bevat de lijst met transacties slechts \'e\'en element bij een \textbf{write lock} of meerdere bij een \textbf{read lock}.

\paragraph{Regels Read/Write Locks}
\begin{enumerate}
	\item Een transactie moet een operatie \textit{read\_lock(X)} of \textit{write\_lock(X)} uitvoeren alvorens een operatie \textit{read\_item(X)} uit te voeren.
	\item Een transactie moet een operatie \textit{write\_lock(X)} uitvoeren alvorens een operatie \textit{write\_item(X)} uit te voeren.
	\item Een transactie moet een operatie \textit{unlock\_item(x)} uitvoeren nadat alle \textit{read\_item(X)} of \textit{write\_item(X)} operaties zijn afgerond.
	\item Een transactie mag geen operatie \textit{write\_lock(X)} uitvoeren wanneer deze al een read lock of write lock bezit. (Dit kan worden versoepeld)
	\item Een transactie mag geen operatie \textit{unlock\_item(X)} uitvoeren wanneer hij de lock niet bezit.
\end{enumerate}

\subsubsection{Conversion of Locks}
We kunnen regel 4 en 5 versoepelen om \textbf{lock conversion} toe te laten. Hierdoor kan de transactie zijn lock
\begin{itemize}
	\item \textbf{Upgraden:} Wanneer deze een read lock bezit kan hij deze \textit{upgraden} tot een write lock.
	\item \textbf{Downgraden:} Wanneer deze een write lock bezit kan hij deze \textit{downgraden} tot een read lock.
\end{itemize}

\subsubsection{Serializability}
\begin{enumerate}
	\item Werkwijze garandeert geen \textbf{serializability}
	\item Protocol nodig dat bepaalt wanneer we locken en unlocken
\end{enumerate}
(zie voorbeeld pagina 761)


\subsection{Guaranteeing Serializability by Two-Phase Locking}% 2
Om te voldoen aan het \textbf{two-phase locking protocol} moeten alle locks voorafgaan aan de \textit{eerste} unlock-operatie. Hiermee kan een transactie worden opgedeeld in twee fases:
\begin{enumerate}
	\item \textbf{Expanding / Growing Phase} \\ Locks kunnen worden bemachtigd maar niet worden vrijgegeven. Bij lock conversion mag worden geupgrade.
	\item \textbf{Shrinking Phase} \\ Locks kunnen worden vrijgegeven maar niet worden bemachtigd. Bij lock conversion mag worden gedowngrade.
\end{enumerate}

\begin{center}
\textbf{Indien elke transactie het two-phase locking protocol volgt is de planning gegarandeerd serializable.}
\end{center}

\paragraph{Eigenschappen}
\begin{itemize}
	\item Limiteert de hoeveelheid parallelisme
	\item Garandeert serializability
	\item Mogelijk zijn er andere schema's die ook voldoen, maar niet worden toegelaten met dit principe
\end{itemize}

\subsubsection{Basic, Conservative, Strict, and Rigorous Two-Phase Locking}
Verschillende varianten op \textbf{Two-Phase Locking}
\begin{itemize}
	\item \textbf{Basic} \\
	Beschreven in vorige paragraaf
	\item \textbf{Conservative of static}
	\begin{itemize}
		\item Vereist dat alles gelockt is alvorent de transactie van start gaat
		\item \textbf{Predeclareren} van een \textbf{write-set} en \textbf{read-set}
		\item Geeft nooit aanleiding tot \textbf{deadlock}
		\item Soms onmogelijk
		\end{itemize}
	\item \textbf{Strict}
		\begin{itemize}
		\item Meest populair
		\item Garandeert stricte schema's
		\item Vrijgeven van exclusieve locks gebeurt pas helemaal op het einde
		\end{itemize}
	\item \textbf{Regorous}
		\begin{itemize}
		\item Nog stricter
		\item Garandeert eveneens stricte schema's 
		\item Vrijgeven van \textit{elke} lock gebeurt pas helemaal op het einde
		\item Eenvoudiger te implementeren dan \textbf{strict}
		\end{itemize}	
\end{itemize}

\subsubsection{Concurrency Control Subsystem}
\begin{itemize}
	\item Verantwoordelijk voor toekennen locks
	\item Houdt gegevensstructuren bij
\end{itemize}


\subsection{Dealing with Deadlock and Starvation}% 2.75
Twee mogelijke problemen die voorkomen bij het gebruik van locks.

\subsubsection{Deadlock}
Treedt op wanneer elke transactie $T$ in een set van minstens twee transacties wacht op een item dat locked is door een andere transactie $T'$
\begin{itemize}
	\item Elke transactie in de wachtrij
	\item Lock kan nooit worden vrijgegeven
\end{itemize}

\subsubsection{Deadlock Prevention Protocols}
Protocols voor het verhinderen van deadlocks.
\begin{itemize}
	\item \textbf{Lock in advance} \\ Zoals gebruikt bij conservative locking
	\item \textbf{Ordering} \\ Bepaal een soort van ordening zodat transacties die meerdere items nodig hebben ze in de juiste volgorde locken.
\end{itemize}
\paragraph{Timestamp Methodes} Alternatieve methodes maken gebruik van een Transaction Timestamp, genoteerd als $TS(T)$. \
\begin{itemize}
	\item Kent uniek ID toe aan elke transactie
	\item Gebaseerd op orde van transactie
	\item $TS(T_1) < TS(T_2) \Rightarrow T_1$ startte voor $T_2$
\end{itemize}
Hiermee kunnen volgende regels worden ingesteld ter preventie van deadlocks.
\begin{itemize}
	\item \textbf{Wait-die} \\ Als $TS(T_i) < TS(T_j)$ dan mag $T_i$ wachten. Indien $TS(T_i) > TS(T_j)$ wordt $T_i$ onderbroken en later herstart met \underline{zelfde timestamp}.
	\item \textbf{Wound-wait} \\ Als $TS(T_i) < TS(T_j)$ dan wordt $T_j$ onderbroken en later herstart met \underline{zelfde timestamp}. Indien $TS(T_i) > TS(T_j)$ mag $T_i$ wachten.
\end{itemize}
Met deze protocols worden geen circulaire wachtrijen gevormd en zijn deadlocks niet mogelijk.
\paragraph{Wait Methods}
\begin{itemize}
	\item \textbf{No Waiting}
	\begin{itemize}
		\item Transactie wordt onmiddellijk onderbroken als geen lock kan worden bemachtigd
		\item Geen controle op daadwerkelijke deadlock
		\item Herstart na zekere periode
	\end{itemize}
	\item \textbf{Cautious Waiting}
	\begin{itemize}
		\item $T_i$ wilt lock voor \textbf{X} maar deze is in het bezit van $T_j$
		\item Als $T_j$ op dat moment zelf ook wacht wordt $T_i$ onderbroken
	\end{itemize}	
\end{itemize}
Beide methodes garanderen dat er geen deadlocks optreden.

\subsubsection{Deadlock Detection}
Protocols voor het opmerken van deadlocks.
\begin{itemize}
	\item Interessant indien kans op deadlocks klein is
	\item Minder interessant indien hoge intensiteit transacties
\end{itemize}
\paragraph{Wait-for graph}
\begin{itemize}
	\item Knoop voor elke transactie die aan het uitvoeren is
	\item Gerichte boog van wachtende transactie naar transactie die lock momenteel heeft
	\item Boog wordt verwijderd indien lock wordt vrijgegeven door eindpunt
\end{itemize}
In geval van deadlock $\rightarrow$ \textbf{vinctim selection}

\subsubsection{Timeouts} 
\begin{itemize}
	\item Transactie gelimiteerd in tijd
	\item Eenvoudig te implementeren
	\item Weinig overhead
\end{itemize}

\subsubsection{Starvation}
\begin{itemize}
	\item Transactie moet wachten voor onbepaalde duur terwijl andere wel kunnen uitvoeren
	\item Treedt op bij prioriteiten (bv ter preventie van deadlocks)
	\item Treedt op wanneer steeds hetzelfde victim wordt gekozen.
\end{itemize}
Hoe oplossen?
\begin{itemize}
	\item Eerlijk wachtschema ($\rightarrow$ Firs-come-first-served)
	\item Prioriteit verhogen met wachtduur
\end{itemize}
Wait-Die en Wound-Wait kunnen geen aanleiding geven tot starvation gezien de transactie herstart wordt met de originele timestamp.



\section{Concurrency Control Based on Timestamp Ordering}% .25
Alternatief dat eveneens \textbf{serializability} garandeert.


\subsection{Timestamps}% .5
\begin{itemize}
	\item Unieke identifier, toegekend bij begin van transactie
	\item Kan gezien worden als de \textbf{starttijd} van de transactie
	\item Ordening op basis van \textbf{timestamps} voorkomt deadlocks
\end{itemize}
Implementatie van een timestamp
\begin{itemize}
	\item Teller die wordt verhoogd bij aanmaak nieuwe timestamp \\
	(Waarde kan niet oneindig toenemen!)
	\item Huidige tijd waarbij slechts 1 timestamp gegenereeerd kan worden tijdens 1 klik van de klok
\end{itemize}


\subsection{The Timestamp Ordering Algorithm}% 2
Centraal idee: \textbf{Sorteren transacties op hun timestamp}
Verschil met 2PL
\begin{itemize}
	\item Schema is equivalent met \textit{een} schema dat voldoet aan 2PL protocol
	\item Schema is equivalent met schema met \textbf{dezelfde volgorde} van transacties 
\end{itemize}
Het algoritme garandeert dat voor elk item dat wordt benaderd door conflicterende operaties de volgorde van benadering gelijk is als die van de timestamps.
Met elk item worden twee Timestamp-waarden geassocieerd:
\begin{itemize}
	\item \textbf{Read Timestamp:} read\_TS(X) \\
	read\_TS(X) $\leftarrow$ TS(T) waarbij T de \textbf{jongste transactie} is die X heeft gelezen.
	
	\item \textbf{Write Timestamp:} write\_TS(X) \\
	write\_TS(X) $\leftarrow$ TS(T) waarbij T de \textbf{jongste transactie} is die X heeft weggeschreven.
\end{itemize}

\subsubsection{Basic Timestamp Ordering (TO)}
\begin{itemize}
	\item 
Transactie T probeert een operatie \textbf{write\_item(X)} uit te voeren. 	Algoritme controleert of de timestamp van T kleiner is dan die van X: 
	\textbf{Boolean b $\leftarrow$ write\_TS(X) $>$ TS(T) $\vee$ read\_TS(X) $>$ TS(X)} \
	\begin{itemize}
		\item \textbf{If(b=TRUE)}: Jongere operatie (met grote TS(T')$>$TS(T)) heeft waarde reeds veranderd $\rightarrow$ stop de transactie, herstel eventueel gewijzigde items (\textbf{roll back}) en weiger de operatie. 
		\item \textbf{If(b=FALSE)}: Geen probleem $\rightarrow$ voer de operatie uit en zet \ write\_TS(X) op TS(T)
	\end{itemize}
	
	\item 
Transactie T probeert een operatie \textbf{read\_item(X)} uit te voeren. 	Algoritme controleert of de timestamp van T kleiner is dan die van X: 
	\textbf{Boolean b $\leftarrow$ write\_TS(X) $>$ TS(T)} \
	\begin{itemize}
		\item \textbf{If(b=TRUE)}: Jongere operatie (met grote TS(T')$>$TS(T)) heeft waarde reeds veranderd $\rightarrow$ stop de transactie, herstel eventueel gewijzigde items (\textbf{roll back}) en weiger de operatie. 
		\item \textbf{If(b=FALSE)}: Geen probleem $\rightarrow$ voer de operatie uit en zet \ read\_TS(X) op $max(TS(T),read\_TS(T))$
	\end{itemize}
\end{itemize}
\textbf{Samengevat:} Wanneer twee conflicterende operaties in verkeerde volgorde worden uitgevoerd, wordt de laatste van de twee geweigerd door het be\"eindigen van bijhorende transactie .

\subsubsection{Strict Timestamp Method}
Variant op Basic TO Algoritme die ganradeert dat de schema's ook \textbf{strict} en \textbf{conflict serializable} zijn.
\\\\
\textbf{Idee:} Transactie T die operatie uitvoert zodat TS(T) $>$ write\_TS(X) \textbf{moet wachten} todat transactie T' dat de huidige waarde van X schreef (write\_TS(X) = TS(T')) voltooid en be\"eindigd is.\\\\
$\rightarrow$ Geen aanleiding tot deadlock: kan niet circulair worden gewacht door ordering in timestamp.

\subsubsection{Thomas's Write Rule}
Variant op Basic TO Algoritme die \textbf{geen conflic serializabilty} garandeert maar minder schrijfoperaties weigert.
\begin{itemize}
	\item \textbf{If(read\_TS(X) $>$ TS(T))}: Stop en herstel (roll back)
	\item \textbf{Else If(write\_TS(X) $>$ TS(T))}: Voer schrijfoperatie \textbf{niet} uit maar blijf uitvoeren. De huidige schrijfoperatie van T is reeds verouderd.
	\item \textbf{Else}: voer de schrijfoperatie uit en zet write\_TS(X) op TS(T).
\end{itemize}



\section{Multiversion Concurrency Control Techniques}% .5
Hierbij worden oudere versies van de data bijgehouden wanneer de waarde van een item wordt gewijzigd. 
\begin{itemize}
	\item Leesoperaties die anders geweigerd zouden worden toch nog een oudere versie lezen.
	\item Door het kiezen van de juiste versie blijft \textbf{serializabilty} gegarandeerd
	\item Vraagt meer ruimte (maar kan soms sowieso het geval zijn voor herstel)
\end{itemize}


\subsection{Multiversion Technique Based on Timestamp Ordering}% .75
Opnieuw twee timestmaps
\begin{itemize}
	\item \textbf{read\_TS($X_i$)}: grootste timestamp (= jongste) van alle transacties die $X_i$ hebben gelezen.
	\item \textbf{write\_TS($X_i$)}: grootste timestamp (= jongste) van alle transacties die $X_i$ hebben weggeschreven.
\end{itemize}
Dit leidt tot 
\begin{itemize}
	\item T mag item $X$ \textbf{schrijven} $\rightarrow$ Versie $X_{k+1}$ van X wordt aangemaakt en read\_TS($X_{k+1}$) en write \_TS($X_{k+1}$) worden op TS(T) gezet.
	\item T mag item $X_i$ \textbf{lezen} $\rightarrow$ read\_TS($X_{i}$) en write \_TS($X_{i}$) worden op TS(T) gezet.
\end{itemize}
\paragraph{Regels} Regels om \textbf{serializability te garanderen:}
\begin{enumerate}
	\item Als transactie T een write\_(X) operatie wil uitvoeren en versie $i$ van X heeft de hoogste write\_TS($X_i$) waarde van alle versies van $X$	 die kleiner of gelijk is aan TS(T) en read\_TS($X_i$) $>$ TS(T)\\ 
	$\rightarrow$ Be\"eindig en herstel transactie T \\
	(Deze situatie komt voor als T een versie van X probeert te schrijven die gelezen zou moeten zijn door een transactie T' met timestamp read\_TS(T'))
	\\ Anders: Vorm een nieuwe versie $X_j$ van X met read\_TS($X_j$) = write\_TS($X_j$) = TS(T)

	\item Als transactie T een read\_(X) operatie wil uitvoeren, vind versie i van X met de hoogste write\_TS($X_i$) van alle X die kleiner zijn van TS(X) \\
	$\rightarrow$ Geef de waarde van $X_i$ terug aan transactie T en zet de waarde van read\_TS($X_i$ op $max(TS(T),read_TS(X_i))$
\end{enumerate}
$\rightarrow$ Lezen is \textbf{altijd} succesvol. \\
$\rightarrow$ Herstellen kan aanleiding geven tot \textbf{cascading rollback}


\subsection{Multiversion Two-Phase Locking Using Certify Locks}% .75
Drie soorten locking
\begin{itemize}
	\item \textbf{Read}:  Onveranderd
	\item \textbf{Write}: Twee versies $\rightarrow$ een transactie T' mag X lezen terwijl T hierop een writ lock bezit.  
	\begin{itemize}
		\item \'E\'en versie weggeschreven door voltooide transactie
		\item Andere versie aangemaakt bij verkrijgen write lock voor T
	\end{itemize}
	\item \textbf{\textit{Certify}}: Exclusief en nodig voor het voltooien van transactie $\rightarrow$ nodig voor elk object waarop write lock is verkregen
\end{itemize}
\paragraph{Eigenschappen}
\begin{itemize}
	\item Lezen kan gelijktijdig met enkelvoudig write lock
	\item Voltooien transactie kan uitgesteld worden door wachten op certify lock
	\item Deadlocks kunnen voorkomen bij het upgraden van read naar write lock
\end{itemize}



\section{Validation (Optimistic) Concurrency Control Techniques}% 1.5
Voorgaande technieken checken \textbf{voor} de uitvoer van de operaties $\rightarrow$ overhead en vertraging.
\\\\
\textbf{Validation} of \textbf{Certification} doet geen controle  tijdens de uitvoer van de transactie.

\paragraph{Protocolfasen}
\begin{enumerate}
	\item \textbf{Read Phase}: Een transactie kan waarden lezen van gecommiteeerde items in de database. Updates worden enkel toegewezen aan lokale kopie\"en.
	\item \textbf{Validation Phase}: Controle wordt uitgevoerd om te verzekeren dat geen updates de \textbf{serializability} schenden.
	\item \textbf{Write Phase:} Als de validatie succesvol was worden de updates weggeschreven naar de database.
\end{enumerate}

\paragraph{Centraal idee}
\begin{itemize}
	\item Controle in \'e\'en keer op het einde
	\item Uitvoer met een minimum aan overhead tot de validatie wordt bereikt	
\end{itemize}

\paragraph{Optimistisch}
\begin{itemize}
	\item Als er veel interferentie is zullen vele transacties moeten herstarten na validatiefase
	\item Optimischtisch veronderstelt weinig interferentie
\end{itemize}

\subsubsection{Protocol}
Maakt gebruik van \textbf{transaction timestamps} en moet daarnaast bijhouden:
\begin{itemize}
	\item write\_sets en read\_sets (= alle items die respectievelijk geschreven en gelezen worden)
	\item Begin- en eindtijden van fases
\end{itemize}
$\rightarrow$ Bij de validatiefase voor $T_i$ wordt nagegaan of deze niet interfereert met gecommiteerde transacties of transacties momenteel ook in de validatiefase zitten.W
\paragraph{Protocol regels}
Tijdens de validatie wordt gecontroleerd of, voor elke transactie $T_j$ dat deze ofwel gecommitteerd is ofwel in zijn validatiefase is geldt:
\begin{enumerate}
	\item Transactie $T_j$ voltooid zijn schrijffase voordat $T_i$ zijn leesfase begint.
	\item $T_i$ start zijn leesfase na de schrijffase van $T_j$ en de read\_set van $T_i$ heeft geen elementen gemeenschappelijk met de write\_set van $T_j$
	\item Zowel de read\_set als de write\_set van $T_i$ hebben geen elementen gemeenschappelijk met de write\_set van $T_j$ en $T_j$ voltooit zijn leesfase $T_i$ zijn leesfase voltooit.
\end{enumerate}



\section{Granularity of Data Items and Multiple Granularity Locking}% .25
We gingen er steeds van uit dat de database gevormd werd door een aantal \textbf{data-items}, bijvoorbeeld:
\begin{itemize}
	\item Een record
	\item Een veld van een record
	\item Een disk block
	\item Volledig bestand
	\item De volledig database
\end{itemize}
De \textbf{granularity} of \textbf{granulariteit} bepaalt de performatie van bovenvermelde technieken.


\subsection{Granularity Level Considerations for Locking}% .75
\textbf{Data item granularity}
\begin{itemize}
	\item \textbf{Fijne granulariteit}: kleine items
	\item \textbf{Grove granulariteit}: grote items 
\end{itemize}
\paragraph{Voor- en nadelen}
\begin{itemize}
	\item \textbf{Grove granulariteit}
	\begin{itemize}
		\item Minder parallellisme
		\item Lange wachttijden
	\end{itemize}
	\item \textbf{Fijne granulariteit}
	\begin{itemize}
		\item Groter aantal \textit{items} in de database
		\item Meer overhead
	\end{itemize}
\end{itemize}
$\rightarrow$ Beste grootte hangt af van situatie


\subsection{Multiple Granularity Level Locking}% 1.75
Ondersteunen van locks voor verschillende groottes van items $\rightarrow$ \textbf{multiple granularity level 2PL}\\
(Probleem: voorbeeld probleem zie pagina 775) \\
Nieuw type lock nodig: \textbf{intention lock} $\rightarrow$ toont pad vanaf root tot gewenste knoop in granulariteitshierarchie.
\begin{enumerate}
	\item \textbf{Itention-shared (IS)}: \'e\'en of meer gedeelde locks op onderliggende kno(o)p(en).
	\item \textbf{Intention-exclusive (IX)} \'e\'en of meer exclusieve locks op onderliggende kno(o)p(en).
	\item \textbf{Shared-intention-exclusive (SIX)} Op huidige knoop rust een gedeeld lock maar \'e\'en of meer exclusieve locks zullen worden aangevraagd op onderliggende knopen.
\end{enumerate}
\paragraph{Protocol regels} \textbf{Multiple Granularity Locking (MLG)}
\begin{enumerate}
	\item Compatibiliteit locks moet aangehouden worden. (zie pagina 776)
	\item De root moet altijd als eerste worden gelockt.
	\item Een knoop $N$ kan gelockt zijn door transactie $T$ in $S$ of in $IS$ mode als en slechts als de vader van de knoop N reeds gelockt is door transactie $T$ in $IS$ ofwel $IX$.
	\item Een knoop $N$ kan gelockt zijn door transactie $T$ in $X$, $IX$ of in $SIX$ mode als en slechts als de vader van de knoop N reeds gelockt is door transactie $T$ in $IX$ ofwel $SIX$.
	\item Een transactie T kan een knoop enkel een lock aanvragen als het geen knopen heeft geunlockt. 
	\item Een transactie T kan een knoop $N$ enkel vrijgegven als geen van de kinderen van $N$ momenteel gelockt zijn door T.
\end{enumerate}
Regel 1 $\rightarrow$ geen conflicten toelaten \\ 
Regel 5,6 $\rightarrow$ forceren 2PL protocol
\paragraph{Toepassing}
Vooral geschikt voor een mix tussen
\begin{itemize}
	\item Korte transacties die slechts een beperk aantal items vereisen
	\item Grote transacties die volledige bestanden vereisen
\end{itemize}



\section{Using Locks for Concurrency Control in Indexes}% 1.25
Locking gebruiken voor indexen (cfr. hoofdstuk 17)\\\\
\textbf{Probleem}: Index begint steeds in root $\rightarrow$ grote hoeveelheid blocking
\begin{itemize}
	\item Root zou gelockt worden in exclusive mode
	\item Alle conflicterende transacties moeten wachten
\end{itemize}
\textbf{Oplossing}: Uitbuiten boomstructuur $\rightarrow$ zoeken via index betekent boom doorlopen
\begin{itemize}
	\item Bovenliggende knoop niet meer nodig eens gepasseerd
	\item Lock op parent mag opgeheven worden eens lock op kind is toegewezen
\end{itemize} 
\paragraph{Conservatieve benadering invoegen}
\begin{enumerate}
	\item Lock root in exclusive mode
	\item Bereik kindknoop
	\item Als het kind niet vol is $\rightarrow$ geef root vrij
\end{enumerate}
$\rightarrow$ Kan toegepast worden op elk niveau van de boom \\
$\rightarrow$ Exlusieve lock worden snel vrijgegeven

\paragraph{Optimistische benadering}
\begin{itemize}
	\item Shared lock voor de knopen leidend naar doelknoop
	\item Exclusive lock op bladknoop
	\item Indien invoeging leidt tot splitsing $\rightarrow$ propageer opwaarts $\rightarrow$ upgrade locks naar exclusive locks
\end{itemize}

\paragraph{Benadering met variant $B^+$-tree: B-link tree}
Hierbij zijn alle broderknopen op hetzelfde niveau gelinkt.
\begin{itemize}
	\item \textbf{Zoeken}
		\begin{itemize}
			\item Shared lock bij opvragen pagina
			\item Lock moet vrijgegeven worden alsvorens kind te benaderen		
		\end{itemize}
	\item \textbf{Invoegen}
	\begin{itemize}
		\item Upgrade shared lock tot exclusive lock
		\item Bij splitsing moet vaderknoop opnieuw gelockt worden in exclusive mode
	\end{itemize}
	\item \textbf{Complicatie}
		\begin{itemize}
			\item Veronderstel invoegoperatie volgt zelfde pad als zoekoperatie
			\item Wordt nieuw item ingevoegd in bladknoop die zorgt voor een splitsing
			\item Wanneer zoektoch daarna wordt voorgezet wordt een foute bladknoop gevonden
			\item Kan toch gevonden worden door \textbf{link} te volgen naar broederknoop
		\end{itemize}
\end{itemize}



\section{Other Concurrency Control Issues}% .25
\subsection{Insertion, Deletion, and Phantom Records}% 1
\subsubsection{Insertion}
Nieuwe record dat wordt ingevoerd $\rightarrow$ kan uiteraard niet benderd worden tot het bestaat
\begin{itemize}
	\item \textbf{Locking environment:} Write lock (exlusief)
	\item \textbf{Timestmap environment:} Timestamp $=$ die van transactie die aanmaakt 
\end{itemize}

\subsubsection{Deletion}
\begin{itemize}
	\item \textbf{Locking environment:} Write lock (exlusief)
	\item \textbf{Timestmap environment:} Latere transacties mogen item niet schrijven of lezen alvorens het verwijderd is
\end{itemize}

\subsubsection{Phantom Records}
Veronderstel volgende situatie
\begin{itemize}
	\item \textbf{Transactie T}: voegt record toe met Dno=5
	\item \textbf{Transactie T'}: selecteert alle records met Dno=5 
\end{itemize}
Mogelijke volgorde van uitvoer:
\begin{itemize}
	\item \textbf{T' na T}: Moet nieuwe record omvatten
	\item \textbf{T na T'}: Mag nieuwe record \textbf{niet} omvatten 
\end{itemize}
Als T' alle records met Dno=5 al heeft gelockt zijn er geen gemeenschappelijke records tussen beide transacties $\rightarrow$ nieuwe record is \textbf{phantom record}

\paragraph{Index Locking} Oplossing voor \textbf{phantom records}
\begin{itemize}
	\item Locken van index alvorens invoegen record
	\item Conflict index lock $\rightarrow$ phantom record wordt opgemerkt
\end{itemize}


\subsection{Interactive Transactions}% .25
Lezen van invoer en schrijven van uitvoer naar een \textbf{interactief apparaat} alvorens gegevens gecommit zijn.
\begin{itemize}
	\item Eventuele afhankelijkheid tussen transacties is niet gekend door systeem
	\item Afhankelijkheid kan niet worden afgedwongen door concurrency control
\end{itemize}
$\rightarrow$ Uitstellen van weergeven tot alles is gecommit.


\subsection{Latches}% .25
Locks die slechts \textbf{voor korte duur} bestaan.
\begin{itemize}
	\item Volgens niet standaard protocol	
	\item Bijvoorbeeld bij wegschrijven pagina naar schijf
\end{itemize}