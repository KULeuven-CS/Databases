\chapter{Introduction to Database Recovery Protocols}
\section{Recovery Concepts}
\subsection{Recovery Outline and Categorization of Recovery Algorithms}
Recovery from transaction 
\begin{itemize}
	\item Restore to the most recent \textbf{consistent} state
	\item Information is kept by \textbf{system log}
\end{itemize}
\subsubsection{Typical Strategy for Recovery}
\begin{enumerate}
	\item Extensive damage due to catastrophic failure
	\begin{itemize}
		\item Recover form last archived backup
		\item Reconstruct current state  by redoing operations of commited transaction
		\item Commited transactions are kept by \textbf{backed up log}
	\end{itemize}
	\item No physical damage is done
	\begin{itemize}
		\item Identify changes causing inconsistencies
		\item Undo transactions that weren't committed
		\item Redo transactions that were committed but not properly finished 
		\\	  (e.g. not yet written to disk)
	\end{itemize}
\end{enumerate}

\subsubsection{Deffered Update}
No physical update on disk until \textbf{after} a transaction is complete.
\begin{itemize}
	\item Updates are recorded by database until committed
	\item Either in main memory or transaction workspace
	\item Written to disk once committed
\end{itemize}
Failures before commit do no harm 
\begin{itemize}
	\item no UNDO necessary
	\item REDO might be necessary:  writing may not have finished after commit
\end{itemize}
$\rightarrow$ \textbf{NO-UNDO/REDO Algorithm} 
\subsubsection{Immediate Update}
Operation may update the database on disk \textbf{before} a transaction is complete.
\begin{itemize}
	\item Operations have to write log on disk before executing
	\item In case of failure before committed: UNDO from log on disk (Rollback)
	\item Both UNDO and REDO
\end{itemize}
$\rightarrow$ \textbf{UNDO/REDO Algorithm}
\subsubsection{UNDO/REDO Operations}
\begin{itemize}
	\item Required to be \textbf{idempotent}
	\item Executing an operations \textbf{multiple times} has to be equivalent to executing it \textbf{once}
	\item Whole recovery process should be \textbf{idempotent}
\end{itemize}
$\rightarrow$ The result of recovery from a system crash during recovery should be the same as the result of recovering when there is no crash during recovery!

\subsection{Caching (Buffering) of Disk Blocks}
Buffering
\begin{itemize}
	\item Disk pages are kept in main memory
	\item Updates are written to copies in memory instead onto disk
	\item Usually handled by OS, but database do it themselves
\end{itemize}
\subsubsection{Directory}
\begin{itemize}
	\item In-memory buffers controlled by DBMS
	\item Directoy is used to keep track of what's in those buffers
	\item If requested file is not in the cache, fetch in from disk and copy it cache
	\item If cache is full
	\begin{itemize}
		\item \textbf{Replace} a page
		\item \textbf{Flush} a cache buffer
	\end{itemize}
	\item Page replacement strategy used
	\begin{itemize}
		\item Least recently used (LRU)
		\item First-in-first-out (FIFO)
		\item Least-likely-to use
	\end{itemize}
\end{itemize}
Additional information in directory
\begin{itemize}
	\item Dirty bit
	\begin{itemize}
		\item Indicates whether of not buffer has been modified
		\item Possibly additional information (such as modifying transaction)
		\item Only pages with dirty bit = 0 have to be written do disk when flushed
	\end{itemize}
	\item Pin-unpin bit
	\begin{itemize}
		\item Pinned page cannot be written back to disk
		\item May be necessary for implementing recovery protocol
	\end{itemize}
\end{itemize}
\subsubsection{Strategies for Flushing}
\begin{itemize}
	\item \textbf{In-place updating}
	\begin{itemize}
		\item Writes buffer to same original location on disk
		\item Overwrites the old value of changed data
	\end{itemize}
	\item \textbf{Shadowing}
	\begin{itemize}
		\item Different disk location
		\item Multiple versions of same data exists
		\item Not typically used in practice
	\end{itemize}
\end{itemize}
We have to types of values
\begin{itemize}
	\item \textbf{Before image} (BFIM): value before updating
	\item \textbf{After image} (AFMI): value after updating
\end{itemize}
\subsection{Write-Ahead Logging, Steal/No-Steal, and Force/No-Foce}
\subsubsection{Write-Ahead Logging}
\begin{itemize}
	\item Use of a log is necessary for in-place updating
	\item BFIM must be recorded in log entry and flushed to disk
	\item BFIM is flushed before AFIM is written to disk
\end{itemize}
$=$ \textbf{Write-Ahead Logging}
\paragraph{Log entries}Distinguish between two types of log entry information
\begin{itemize}
	\item \textbf{REDO-type log}
	\begin{itemize}
		\item Includes the \textbf{new value} (AFIM)
		\item Needed to \textbf{redo} the operation
	\end{itemize}
	\item \textbf{UNDO-type log}
	\begin{itemize}
		\item Includes the \textbf{old value} (BFIM)
		\item Needed to \textbf{undo} the operation
	\end{itemize}	
\end{itemize}
$\rightarrow$ Combined for \textbf{UNDO/REDO-Algorithm}
\paragraph{Log Buffering}
\begin{itemize}
	\item Log is stored in log buffer in main memory
	\item Log consists of append-only file
	\item Last $n$ block of that file are in cache
\end{itemize}
$\rightarrow$ Must first be written to disk \textbf{before} the data block itself can be written back!
\subsubsection{Steal/No-steal}
\begin{itemize}
	\item \textbf{No-steal approach}
	\begin{itemize}
		\item Cache buffer page update cannot be written to disk before transaction commits
		\item Indicated by pin-bit
	\end{itemize}
	\item \textbf{Steal approach}
	\begin{itemize}
		\item Allows writing an updated buffer before the transaction commits
		\item Used when buffer manager needs a buffer frame and replaces an existing page that had been updated but whose transaction had not committed.
	\end{itemize}
\end{itemize}
\subsubsection{Force/No-force}
\begin{itemize}
	\item \textbf{Force approach}
	\begin{itemize}
		\item Updated pages are written to disk before transaction commits
		\item REDO will never be needed
	\end{itemize}
	\item \textbf{No-force approach}
	\begin{itemize}
		\item The opposite, obviously
	\end{itemize}
\end{itemize}
Typically a \textbf{steal/no-force} approach is used.
\begin{itemize}
	\item Advantage is a page of a committed transaction may still be in the buffer
	\item Eliminates expensive IO
\end{itemize}
\subsubsection{Recovery Lists}
To facilitate recovery: the DBMS may need to maintain a number of lists
\begin{itemize}
	\item \textbf{Active transactions}: have started but not committed
	\item \textbf{Committed transactions}
	\item \textbf{Aborted transactions}: since last checkpoint
\end{itemize}
$\rightarrow$ Maintaining those lists makes recovery more efficient!
\subsection{Checkpoints in the System Log and Fuzzy Checkpoints}
\subsubsection{Checkpoints}
A checkpoint is a \textbf{list of active transactions	} that is periodically written into the log when all changed buffers are written to disk.
\begin{itemize}
	\item 	All transactions have their commit-entries in the log before a checkpoint
	\item Do not have to have their write operation redone after checkpoint
	\item List of active transactions is included in checkpoint $\rightarrow$ easily redone
\end{itemize}
DBMS decides on what intervals to take a checkpoint. This consists of:
\begin{enumerate}
	\item Suspending execution of transactions temporarily
	\item Force-writing all changed main memory buffers to disk
	\item Writing checkpoint-record to log and force-writing it to disk
	\item Resuming execution
\end{enumerate}
$\rightarrow$ Additional info may be included, facilitating undoing in case of a rollback.
\subsubsection{Fuzzy Checkpoints}
Transactions may be delayed due to checkpoint taking $\rightarrow$ \textbf{fuzzy checkpointing} reduces this.
\begin{itemize}
	\item Transactions resumes after writing of begin-checkpoint
	\item After step 2 $\rightarrow$ write end-checkpoint to disk
	\item Previous checkpoint should remain valid
	\item Pointer on disk points to previous checkpoint until end-checkpoint is reached
\end{itemize}
\subsection{Transaction Rollback and Cascading Roll back}
A \textbf{rollback} is needed when a transaction fails before committing. Changed data must be changed back to its \textbf{original} state.
\subsubsection{Cascading Rollback}
\begin{enumerate}
	\item Transaction T fails and rolls back
	\item Transaction S had read data changed by T \\ $\rightarrow$ S has to be rolled back also.
\end{enumerate}
This can process can continue for a long time and is called \textbf{cascading roll back}.
This can be \textbf{complex} and \textbf{time consuming} $\rightarrow$ systems are designed so that this is never needed.
\\
For an example: check page 792 in Database Systems (Sixth Edition)
\subsection{Transactions Actions That Do Not Affect the Database}
In general, there are actions that do not affect the database
\begin{itemize}
	\item Should not execute multiple times in case of recovery
	\item Execute after transaction committed
	\item Create a batch job which is executed afterwards
\end{itemize}
\section{NO-UNDO/REDO Recovery Base on Deffered Update}
$\rightarrow$ Postpone updates until the transaction completes and reaches its commit point.
\begin{itemize}
	\item Updates are recorded to a log
	\item Log is force-written to disk when committing
	\item Only REDO-entries needed
	\item Only applicable if transactions change few items
\end{itemize}
$\rightarrow$ Buffers will run out of space!
\subsubsection{Protocol}
\begin{enumerate}
	\item A transaction cannot change the database on disk until it reaches its commit point.
	\item A transaction does not reach its commit point untill all its REDO-type log entries are recorded in the log \textbf{and} the log buffer is written to disk.
\end{enumerate}
Same properties as listed above.
\subsubsection{Multiuser Systems} Concurrency control and recovery are interrelated. Consider a system with concurrency control using 2PL. We call a possible algorithm \textbf{RDU\_M} or Recovery used Deferred Update in a Multiuser environment. \\
\\
\textbf{Procedure RDU\_M (NO-UNDO/REDO with checkpoints)}
\begin{itemize}
	\item Two lists of transactions
	\begin{itemize}
		\item Committed transaction T since last checkpoint: \textbf{commit list}
		\item Active transactions T': \textbf{active list}
	\end{itemize}
	\item REDO all write operations of the committed transaction list from the log, \textbf{in the order they were written to the log}
	\item Active transactions did not commit hence no action is required
\end{itemize}
\textbf{Procedure REDO (WRITE\_OP)}
\begin{itemize}
	\item Examin the log entry
	\item Set item to the AFIM
\end{itemize}
For an example: check page 794 in Database Systems (Sixth Edition)
\subsubsection{Efficiency improvements}
If a data item X has been updated more than once by committed transactions since the last checkpoint, it is only necessary to REDO the \textbf{last update of X} from the log during recovery! \\
$\rightarrow$ Start from the end of the log
\subsubsection{Aborted Transactions}
If a transaction is aborted it is simply resubmitted since it has not changed the database. This has drawbacks:
\begin{itemize}
	\item Limits concurrent execution
	\item All write-locked items remain locked until commit-point is reached
	\item Requires excessive buffer space to hold items
\end{itemize}
The main benefit is that a transaction operation \textbf{never has to be undone} for two reasons:
\begin{enumerate}
	\item A transaction does not record any changes until commit-point $\rightarrow$ never rolled back
	\item A transaction will never read the value of an item that is written by an uncommitted transaction
\end{enumerate}
\section{Recovery Rechniques Based on Immediate Update}
$\rightarrow$ Updates are \textbf{immediately} written to disk. Though this is not a requirement!
\subsubsection{Handling Failed Transactions}
\begin{itemize}
	\item Undoing the effect of updates
	\item Rolling back transactions
	\item UNDO-type log entries are used which contain the \textbf{old value} (BFIM)
	\item Require \textbf{steal strategy}
\end{itemize}
We can distinguish two main types of algorithms:
\begin{itemize}
	\item Updates recorded on disk before transaction commits
	\begin{itemize}
		\item Never need to REDO
		\item Called \textbf{UNDO/NO-Redo recovery algorithm}
		\item Used Force-Strategy for deciding when buffers are written to disk
	\end{itemize}
	\item Transaction is allowed to commit before updates are written to disk
	\begin{itemize}
		\item \textbf{UNDO/REDO recovery algorithm}
		\item \textbf{Steal/No-force} strategy
	\end{itemize}
\end{itemize}
\subsubsection{Concurrent execution}
Recovery process depends on concurrency control protocol
\begin{itemize}
	\item RIU\_M: Recovery using Immediate Updates for a Multiuser environment
	\item We assume checkpoints and a strict schedules
	\item Deadlocks may occur thus requires abort and UNDO-operations
\end{itemize}
\textbf{Procedure RIU\_M (UNDO/REDO with checkpoints)}
\begin{enumerate}
	\item Use two lists maintained by the system
	\begin{itemize}
		\item Committed transactions
		\item Active transactions
	\end{itemize}
	\item Undo all write\_item operations of \textbf{active, uncommitted} transactions using UNDO (in reverse order of log)
	\item Redo all write\_item operations of \textbf{committed} transactions from the log in regular order
\end{enumerate}
Whenever an operation is redone, it's added to a list so it is not redone again.
\\\\
\textbf{Procedure REDO (Write\_OP)}
\begin{enumerate}
	\item Examine log entry and set value of X back to old value (BFIM)
	\item Undo write\_item in \textbf{reverse} order
\end{enumerate}
Only one UNDO operation is needed $\rightarrow$ scan list in forward direction and apply earliest UNDO.

\section{Shadow Paging}
\begin{itemize}
	\item Does not require use of a log in single-user environment
	\item Log may be used in multiuser environment
\end{itemize}
\subsection{Before Transaction}
\begin{itemize}
	\item Database consists of  fixed size pages
	\item Directory with $n$ entries with $i$th entry the $i$th page on disk
	\item Directory is kept in main memory
	\item When transaction begins executing the \textbf{current directory} is copied into a \textbf{shadow directory}
	\item Shadow directory is saved on disk while active is used
\end{itemize}

\subsection{During Transaction}
\begin{itemize}
	\item Shadow directory is not touched during transaction
	\item In case of write\_item, new page is created
	\item Old page is \textbf{not} overwritten
	\item Current directory points to new page, whereas the old one points to the previous
	\item 
\end{itemize}
$\rightarrow$ For every updates page, there are \textbf{two} version. One referenced to by the shadow directory, and one by the active directory.

\subsection{Recovery Process}
\begin{itemize}
	\item Free the modified database page $\rightarrow$ discard current directory
	\item Previous state is available trough the shadow directory
	\item Committing corresponds to discarding shadow directory
\end{itemize}
$\rightarrow$ No UNDO/REDO necessary!

\subsection{Disadvantages}
\begin{itemize}
	\item Logs and checkpoints need to be incorporated into page for multiuser envornment
	\item Changing location on disk $\rightarrow$ less efficient
	\item Large directories may exist
	\item Overhead of writing shadow directories to disk
	\item Old page must be released $\rightarrow$ Garbage collection
	\item Migration between directories has to be \textbf{atomic}!
\end{itemize}

\section{The Aries Recovery Algorithm}
\begin{itemize}
	\item Used in database related IBM products
	\item Uses a \textbf{steal/no-force} approach
\end{itemize}
Based on 3 concepts:
\begin{enumerate}
	\item Write-ahead logging
	\item Repeating history during REDO
	\begin{itemize}
		\item Retrace all actions prior to the crash
		\item Uncommitted transactions at the time of the crash are omitted
	\end{itemize}
	\item Logging changes during UNDO
	\begin{itemize}
		\item Prevents from repeating  the completed UNDO operations
		\item In case of restart during recovery
	\end{itemize}
\end{enumerate}

\subsection{Analysis Phase}
\begin{itemize}
	\item Identifies dirty buffer pages in buffer
	\item Appropriate location from where REDO should start is determined
\end{itemize}

\subsection{REDO Phase}
\begin{itemize}
	\item Reads start point for REDO
	\item From there REDO operations are applied until end of log is reached
	\item Only necessary REDO operations are applied
\end{itemize}

\subsection{UNDO Phase}
\begin{itemize}
	\item Log is scanned backwards
	\item Operations are undone in reverse order
\end{itemize}

\subsection{Structures}
\begin{itemize}
	\item Log
	\item Transaction Table
	\item Dirty Page Table
	\item Checkpoints
\end{itemize}

\subsection{Log Sequence Number (LSN)}
\begin{itemize}
	\item Monotonically increasing
	\item Indicates the address of the log record on disk
	\item Corresponds to specific change of some transaction
	\item Each page will store LSN of the \textbf{last record corresponding to a change for that page}
	\item Log is written when
	\begin{itemize}
		\item Updating a page (write)
		\item Committing a transaction (commit)
		\item Aborting a transaction (abort)
		\item Undoing an update (undo) $\rightarrow$ compensation log record
		\item Ending a transaction (end)$\rightarrow$ end log record
	\end{itemize}
\end{itemize}
Fields in log
\begin{itemize}
	\item \textbf{Previous LSN}: Links the log records in reverse order for each transaction
	\item For write record
	\begin{itemize}
		\item Page ID
		\item Length of the updated item
		\item Offset from the beginning of the page
	\end{itemize}
\end{itemize}
\subsection{Tables}
\begin{itemize}
	\item Transaction Table
	\begin{itemize}
		\item Entry for each active transaction
		\item Information such as ID, status and the LSN of the most recent log record
	\end{itemize}
	\item Dirty Page Table
	\begin{itemize}
		\item Entry for each dirty page in buffer
		\item Includes page ID, LSN of the earliest update to that page
	\end{itemize}
\end{itemize}
$\rightarrow$ Mainly used by transaction manager

\subsection{Checkpoints}
\begin{itemize}
	\item Writing a begin\_checkpoint record
	\item Writing an end\_checkpoint record
	\begin{itemize}
		\item Content of both the transaction table and Dirty Page Table are appended tot the log
		\item Fuzzy checkpointing is used
	\end{itemize}
	\item Writing LSN of the begin\_checkpoint to a special file
	\begin{itemize}
		\item Accessed during recovery
		\item Used to locate last checkpoint information
	\end{itemize}
\end{itemize}

\subsection{Recovery from crash}
\begin{itemize}
	\item Information from the last checkpoint is first accessed trough the special file 
	\item 
\end{itemize}

\textbf{For the detailed exaplanation of the ARIES Algorithm, refer to page 799 in Database Concepts (6th edition)}

\section{Recovery in Multidatabse Systems}
\begin{itemize}
	\item Single transaction may need acces to multiple databases
	\item May even be different types of databases
	\item Recovery techniques may differ between databases
\end{itemize}
$\rightarrow$ \textbf{Global Recovery Manager} of \textbf{Coordinator} is needed. \\
$\rightarrow$ Follows \textbf{two-phase commit protocol}
\begin{itemize}
	\item \textbf{Phase 1} 
	\begin{itemize}
		\item All participation database signal coordinator that their part has concluded
		\item Coordinator sends message \textbf{prepare for commit}
		\item Every receiving database will force-write all log records etc to disk
		\item Answer coordinator with \textbf{ready to commit}
		\item If write fails send cordinator \textbf{cannot commit}
		\item Assumes a \textbf{not OK} reply if no response after a time interval
	\end{itemize}
	\item \textbf{Phase 2}
	\begin{itemize}
		\item All participants reply OK
		\item Coordinators vote is also OK
		\item Transaction is successful
		\item Coordinator sends a \textbf{commit} signal
		\item In case of failure: roll back message
	\end{itemize}
\end{itemize}
$\rightarrow$ Either \textbf{all} databases commit or \textbf{all} database roll back.
\section{Database Backup and Recovery from Catastrophic Failures}
\begin{itemize}
	\item Techniques so far focussed on \textbf{non-catastrophic} failures
	\item Entries from log or shadow directories are used
\end{itemize}
\subsection{Handling more Catastrophic Failures}
\begin{itemize}
	\item Databse Backup
	\item Whole database and log are periodically copied to cheap storage medium
	\item The latest backup can be reloaded
	\item Such backups may be locked in a vault for critical information
	\begin{itemize}
		\item Stock markets
		\item Banking and insurance
	\end{itemize}
\end{itemize}
\subsection{Increase frequency}
\begin{itemize}
	\item Create customary backups
	\item Start where last complete backup ended
	\item Smaller thus possible to create more frequently
	\item After full recovery, redo operations noted in this backup
\end{itemize}


